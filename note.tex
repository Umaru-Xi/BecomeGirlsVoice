\include{header}

\begin{document}
 \begin{center} 
 ~\\ ~\\ ~\\ ~\\ ~\\ ~\\ ~\\ ~\\ 
 \Large \textbf{成为女孩子的声音(训练篇)}\\
 \large (笔记)\\ ~\\ ~\\ ~\\ ~\\ ~\\ ~\\ ~\\ ~\\ 
 \large  \textbf{原作}: \mincho 「女の子の声になろう!即効ボイトレ編」\rmfamily \\ \normalsize \textbf{作者}: 白石謙二 \\
 \normalsize \textbf{翻译}: \mincho 佐倉奈緒\rmfamily (知乎用户) \\ \small \textbf{笔记}: Umaru Aya
 ~\\ ~\\ ~\\  ~\\ ~\\ ~\\  ~\\ ~\\ ~\\ ~\\ ~\\ ~\\ ~\\ ~\\ ~\\ ~\\ ~\\ ~\\~\\ ~\\ ~\\ ~\\ ~\\ ~\\ ~\\
 \normalsize \textbf{(编译前请注释掉数据纸)}\\ 
 \small \textbf{编辑器}: Kile\\ \textbf{编译器}: XeLaTex
\end{center}
  
\clearpage

\quad 需要注意: 原书\mincho 「女の子の声になろう!即効ボイトレ編」\rmfamily 版权归作者\mincho 白石謙二\rmfamily 所有. \par
\quad 翻译稿「成为女孩子的声音」版权归知乎用户\mincho 佐倉奈緒\rmfamily 所有. \par
\quad 本副本是 \mincho 埋まる 亜弥 \rmfamily 在学习过程中仿照「数学手册」形式整理而成的笔记. LaTex源码采用BSD协议开放, 但原始代码及其编译成果的版权归笔记作者所有. \par
\quad 由此练习引发的一切后果均需读者自行承担. 原始作者相关内容以原始文件为准.


\clearpage

\large 
\begin{center}
 \textbf{1.为了发出良好的声音进行准备}
\end{center}

[\textbf{大体的练习过程}] \par
\qquad 1. 发声准备: 放松发声器官;\par
\qquad 2. 使用腹式呼吸: 便于控制声音;\par
\qquad 3. 锻炼本声;\par
\qquad 4. 锻炼假声;\par
\qquad 5. 掌握混声: 使本声和假声音域靠近;\par
\qquad 6. 掌握发声要领: 不费力地发声;\par
\qquad 7. 进行会话练习: 发出成型语句;\par
\qquad 8. 进行音色练习: 发出期待的声音.\\

[\textbf{面部的伸展运动}]\par
\qquad 1. 面部肌肉及下颚的伸展: 张大眼睛和嘴巴并发出``啊\textasciitilde''的声音, 然后五官向面部中心集中并发出``嗯\textasciitilde''的声音, 重复8次左右;\par
\qquad 2. 嘴唇的伸展: 嘴唇左右横向大幅度张开发出``依\textasciitilde''的声音, 然后突出嘴唇发出``呜\textasciitilde''的声音, 重复8次左右;\par
\qquad 3. 重复发出: あ、い、う、え、お, 重复8次左右, 逐步加快(声音可以较小);\par
\qquad 4. 重复发出: あい、おあ、いう、うあ、あう、おえ、えお、うえ, 每组重复8次. い、え、あ发音时上下牙齿各露出10颗, う、お发音时不露出牙齿.\\

[\textbf{舌头的伸展运动}]\par
\qquad 1. 交替舔左右脸颊内侧16次左右;\par
\qquad 2. 重复发出: らな、りち、るめ、れさ、ろけ, 重复8次左右, 逐步加快(可以小声).\\

[\textbf{嗓子的伸展运动}]\par
\qquad 1. 张大嘴尝试大哈欠, 重复四次;\par
\qquad 2. 需要注意, 声带的训练可能需要日积月累.\\

[\textbf{肩部的伸展运动}]\par
\qquad 1. 两肩分别向前后各8圈转动;\par
\qquad 2. 两手交叉尽量向前伸展(弯腰), 保持20秒;\par
\qquad 3. 双手向后, 同第二步.\\

\clearpage

[\textbf{脖子的伸展运动}]\par
\qquad 1. 脖子缓慢地转动, 顺时针与逆时针各4圈;\par
\qquad 2. 前后左右倾斜头部. 向后时用双手上压下颚, 向前时用双手向前压. 向左右时用手拖住头部. 每次倾斜保持8秒.\\

[\textbf{练习时的姿势}]\par
\qquad 1. 平稳的站姿: 两脚与肩膀同宽站立, 大多数人会将重心偏向一侧, 不需要很严密只要站稳即可;\par
\qquad 2. 使身体完全固定站直, 可以选择紧贴墙壁的方式进行;\par
\qquad 3. 保持后颈伸直, 视线向前或稍稍向上;\par
\qquad 4. 将手腕放在舒适的位置即可(如自然下垂, 怎么方便怎么来但不要用力伸直).\\

[\textbf{笔记中使用的日语假名记号发音对照}] 括号中为汉语拼音发音与罗马音存在差别时的汉语拼音注音.\par
\qquad あ----a\qquad い----i\qquad う----u\qquad え----e\qquad お----o\par
\qquad か----ka\qquad き----ki\qquad く----ku\qquad け----ke\qquad こ----ko\par
\qquad さ----sa\qquad し----shi(xi)\qquad す----su\qquad せ----se\qquad そ----so\par
\qquad た----ta\qquad ち----chi(qi)\qquad つ----tsu\qquad て----te\qquad と----to\par
\qquad な----na\qquad に----ni\qquad ぬ----nu\qquad ね----ne\qquad の----no\par
\qquad は----ha\qquad ひ----hi\qquad ふ----fu\qquad へ----he\qquad ほ----ho\par
\qquad ま----ma\qquad み----mi\qquad む----mu\qquad め----me\qquad も----mo\par
\qquad や----ya\qquad い----i\qquad ゆ----yu\qquad え----e\qquad よ----yo\par
\qquad ら----ra(la)\qquad り----ri(li)\qquad る----ru(lu)\qquad れ----re(le)\qquad ろ----ro(lo)\par
\qquad わ----wa\qquad い----i\qquad う----u\qquad え----e\qquad を----wo\par
\qquad ん----n(eng)\\

[\textbf{日语假名记号浊音变化}] k-g, s-z, t-d, h-b.

\clearpage

\begin{center}
 \textbf{2.把握自己的声音吧}
\end{center}

[\textbf{关于数据记录}] 笔记结尾附带数据记录纸, 编译前请确认删除敏感数据.\\

[\textbf{确认声音的弱点}]\par
\qquad 1. 测试方法: 用手按住喉结的凸出部分, 向上轻轻推动喉结, 尝试发出``ほ\textasciitilde"的声音, 以及あ、い、う、え、お. 注意不要超过自己的极限, 出现疼痛感请立刻停止;\par
\qquad 2. 可能的情况1: 声音太粗, 像老太婆或接近动画角色的声音;\par
\qquad 3. 可能的情况2: 完全没有女声的感觉, 需要对声音高低自由调节进行训练;\par
\qquad 4. 可能的情况3: 声音根本发不出来或只能发出气息声, 需要首先尝试发出假声.\\

[\textbf{频率测试软件}] \par
\qquad Linux: Audacity;\par
\qquad Android: Vocal Pitch Monitor.\\

[\textbf{基准频率测试}] 用平常的声调发出''あ\textasciitilde"的声音约3秒, 紧接着发出``い"的声音约3秒, 可重复多次.\\

[\textbf{测量最低和最高频率}] 先以平常的声调发出''あ\textasciitilde"的声音约2秒, 停顿1秒后升调或降调. 重复此过程直到无法再升调或降调.\\

[\textbf{参考频率}] 男性80Hz-140Hz, 女性220Hz以上.\\

[\textbf{频率音调对照表(单位:Hz)}]\par
\qquad G5----784.00\qquad F\#/Gb5----739.99\qquad F5----698.46\qquad E5----659.26\par
\qquad D\#/Eb5----622.25\qquad D5----587.33\qquad C\#/Db5----554.37\qquad C5----523.25\par
\qquad B4----493.88\qquad A\#/Bb4----466.16\qquad A4----440\qquad G\#/Ab4----415.31\par
\qquad G4----392.00\qquad F\#/Gb4----396.99\qquad F4----349.23\qquad E4----329.63\par
\qquad D\#/Eb4----311.13\qquad D4----293.67\qquad C\#/Db4----277.18\qquad C4----261.63\par
\qquad B3----246.94\qquad A\#/Bb3----233.08\qquad A3----220.00\qquad G\#/Ab3----207.65\par
\qquad G3----196.00\qquad F\#/Gb3----185.00\qquad F3----174.61\qquad E3----164.81\par
\qquad D\#/Eb3----155.56\qquad D3----146.83\qquad C\#/Db3----138.59\qquad C3----130.81\par
\qquad B2----123.47\qquad A\#/Bb2----116.54\qquad A2----110.00\qquad G\#/Ab2----103.83\par
\qquad G2----98.00\qquad F\#/Gb2----92.50\qquad F2----87.31\qquad E2----82.41\\

[\textbf{音域}] 最高频率减最低频率.

\clearpage

\begin{center}
 \textbf{3.尝试着进行发声练习吧}
\end{center}

[\textbf{腹式呼吸}]\par
\qquad 1. 平躺做出仰卧起坐姿势, 取一本较厚的书(如数学手册或电动力学题解)放在腹部上;\par
\qquad 2. 全身放松, 将肺中的空气吐出, 不要让书本掉落;\par
\qquad 3. 缓慢地吸入空气, 自燃地使腹部膨胀, 不要让书本掉落;\par
\qquad 4. 重复上面的步骤, 直至能够顺利完成练习标准.\\

[\textbf{发声练习1}] 尝试以4拍长度吸气, 吐气8拍. 跟随<音轨1>进行练习.\par
\qquad 1. 呼吸时从鼻子吸入空气, 从嘴中吐出;\par
\qquad 2. 吐气时为了能够使力可以发出轻微的``呼\textasciitilde"声或不发出声音;\par
\qquad 3. 要将空气完全吐出以至于不能再吐;\par
\qquad 4. 本练习需要重复4次.\\

[\textbf{发声练习2}] 尝试以4拍长度吸气, 摒住呼吸4拍, 吐出空气4拍, 完全吐出空气4拍. 跟随<音轨2>进行练习. 重复4次.\par
\qquad 如果CD速度过快, 可根据自身情况进行速度调节.\\

[\textbf{发声练习3}] 尝试以4节拍长度吸气, 摒住呼吸4拍, 分8拍以''呼\textasciitilde"声吐气, 直至将空气完全吐出. 跟随<音轨3>重复练习4次.\\

[\textbf{发声练习4}] 尝试以4拍长度吸气, 摒住呼吸4拍, 让嘴唇颤动着吐出空气(So\#音8拍). 跟随<音轨4>重复练习4次.\par
\qquad 如果不能发出唇颤音, 可以尝试用手指向上推动嘴唇两端稍下位置.\\

[\textbf{发声练习5}] 尝试以4拍吸入空气, 摒住呼吸4拍, 卷起舌头让舌头震动着吐出空气(So\#音8拍). 跟随<音轨5>重复练习4次.\par
\qquad 用卷起的舌尖顶住上颚吐气, 舌头会震动.\\

[\textbf{发声练习6}] 尝试以4拍吸入空气, 摒住呼吸4拍, 发出``ん\textasciitilde"的声音(So\#音8拍). 跟随<音轨6>重复练习4次.\par
\qquad 不要使用嗓子发音, 让声音集中于鼻子的位置.\\

[\textbf{发声练习7}] 尝试以4拍吸入空气, 摒住呼吸4拍, 发出''あ\textasciitilde"的声音(So\#音8拍). 跟随<音轨7>重复练习4次.\par
\qquad 发声中保持嘴纵向张开, 中途不要改变嘴的大小.\\

\clearpage

[\textbf{发声练习8}] 尝试以4拍吸入空气, 摒住呼吸4拍, 发出``ん\textasciitilde"的声音(Do\#音8拍). 跟随<音轨8>重复练习4次.\\

[\textbf{发声练习9}] 尝试以4拍吸入空气, 摒住呼吸4拍, 发出''あ\textasciitilde"的声音(Do\#音8拍). 跟随<音轨9>重复练习4次.\par
\qquad 不要改变面部的朝向并尝试让声音从头顶部发出(震动感). 尝试声音沿喉管的延长线向上发出.\\

[\textbf{发声练习10}] 每空1拍尝试以Mi音发``な"音1拍. 跟随<音轨10>重复练习4次.\par
\qquad 尝试每半音进行音高的提升, 跟随伴奏发出声音.\\

[\textbf{发声练习11}] 不空节拍以Mi音发''な"音. 跟随<音轨11>重复练习4次.\par
\qquad 尝试每半音进行音高的提升, 跟随伴奏发出声音.\\

[\textbf{发声练习12}] 连续地以Mi音半拍发``な"音. 跟随<音轨12>重复练习4次.\par
\qquad 尝试每半音进行音高的提升, 跟随伴奏发出声音.\\

[\textbf{发声练习13}] 以Mi,Fa\#,So\#,Fa\#,Mi的音阶发''ま"音各1拍. 跟随<音轨13>重复练习4次.\par
\qquad 尝试每半音进行音高的提升, 跟随伴奏发出声音.\\

[\textbf{发声练习14}] 以Mi,Fa\#,So\#,Fa\#,Mi的音阶每拍连续发2次''ま"音. 跟随<音轨14>重复练习4次.\par
\qquad 尝试每半音进行音高的提升, 跟随伴奏发出声音.\\

[\textbf{发声练习15}] 以Mi,Fa\#,So\#,Fa\#,Mi的音阶每半拍连续发4次''ま"音. 跟随<音轨15>重复练习4次.\par
\qquad 尝试每半音进行音高的提升, 跟随伴奏发出声音.\\

[\textbf{发声练习16}] 以8拍长度分别以Mi,Fa,Fa\#,So,So\#,La音发``ら"音. 跟随<音轨16>重复练习4次.\par
\qquad 尝试每半音进行音高的提升, 跟随伴奏发出声音.\\

[\textbf{发声练习17}] 以9拍长度分别以Mi,Fa\#,So\#,La,Si,Do\#,Re音找到节奏发``ら"音. 跟随<音轨17>重复练习4次.\par
\qquad 尝试每半音进行音高的提升, 跟随伴奏发出声音.\\

\clearpage

[\textbf{发声练习18}] 以Mi,Fa\#,So\#,Fa\#,Mi,Fa\#,So\#,Fa\#,Mi音发``ま"音共9拍. 跟随<音轨18>重复练习4次.\par
\qquad 尝试每半音进行音高的提升, 跟随伴奏发出声音.\\

[\textbf{发声练习19}] 以Mi音半拍节奏发''られりろ"音. 跟随<音轨19>重复练习4次.\par
\qquad 尝试每半音进行音高的提升, 跟随伴奏发出声音.\\

[\textbf{发声练习20}] 以Mi音连续发''られりろ"音. 跟随<音轨20>重复练习4次.\par
\qquad 尝试每半音进行音高的提升, 跟随伴奏发出声音.\\

[\textbf{发声练习21}] 以Mi,Fa\#,So\#,La,Si,La,So\#,Fa\#音每半拍发``させしすせそさそ"音. 跟随<音轨21>重复练习4次.\par
\qquad 尝试每半音进行音高的提升, 跟随伴奏发出声音.\\

[\textbf{发声练习22}] 以Mi,Fa\#,So\#,La,Si,La,So\#,Fa\#音连续发``させしすせそさそ"音. 跟随<音轨22>重复练习4次.\par
\qquad 尝试每半音进行音高的提升, 跟随伴奏发出声音.\\

\clearpage

\begin{center}
 \textbf{4.尝试着进行假声练习吧}
\end{center}

[\textbf{男女声音差异的原因}] 男性声带长度大约是女性声带长度的1.5倍, 且男性声带位置比女性低.\\

[\textbf{无法发出假声的解决方法}]\par
\qquad 1. 通过手术改变声带长度和位置: 需要考虑之后的人生设计, 同时有不成功的案例;\par
\qquad 2. 通过锻炼声带周围的肌肉: 大部分情况都可以做到.\\

[\textbf{无法发出假声的最可能原因}] 将过多无用的力气集中在嗓子上.\\

[\textbf{自然而然会发出假声的常见条件}]\par
\qquad 1. 全身疲劳, 嗓子没有力气时;\par
\qquad 2. 即使大喊也不会赶到害羞时.\\

[\textbf{假声转变到女声的方法}]\par
\qquad 1. 使用''やほ\textasciitilde"的方式发出假声;\par
\qquad 2. 以这样的状态反复发声;\par
\qquad 3. 尝试降低假声音调进行练习(参考女声频率范围).\\

[\textbf{发声练习23}] 尝试以4拍吸入空气, 摒住呼吸4拍, 以假声方式通过一边让嘴唇颤动一边发So\#音8拍的方式吐净空气. 跟随<音轨23>重复练习4次.\\

[\textbf{发声练习24}] 尝试以4拍吸入空气, 摒住呼吸4拍, 以假声方式通过一边卷起舌头使其震动让一边发So\#音8拍的方式吐净空气. 跟随<音轨24>重复练习4次.\par
\qquad 舌颤音需要一定的技巧, 如果没办法做到可以先跳过这里之后再尝试.\\

[\textbf{发声练习25}] 尝试以4拍吸入空气, 摒住呼吸4拍, 以假声方式So\#音发``は\textasciitilde"共8拍. 跟随<音轨25>重复练习4次.\\

[\textbf{发声练习26}] 尝试以4拍吸入空气, 摒住呼吸4拍, 以假声方式Do\#音发``は\textasciitilde"共8拍. 跟随<音轨26>重复练习4次.\par
\qquad 在口角上扬的同时发声, 感受声音从脸颊向上发出.\\

[\textbf{发声练习27}] 以假声方式每空1拍以Mi音发出''ほ"音1拍. 跟随<音轨27>重复4次.\\

[\textbf{发声练习28}] 以假声方式每1拍以Mi音发出''ほ"音. 跟随<音轨28>重复4次.\par
\qquad 尝试每半音进行音高的提升, 跟随伴奏发出声音.

\clearpage

[\textbf{发声练习29}] 以假声方式连续以Mi音发出''ほ"音各半拍. 跟随<音轨29>重复4次.\par
\qquad 尝试每半音进行音高的提升, 跟随伴奏发出声音.\\

[\textbf{发声练习30}] 以假声方式分别以Mi,Fa\#,So\#,Fa\#,Mi音发``は"音各1拍. 跟随<音轨30>重复4次.\par
\qquad 尝试每半音进行音高的提升, 跟随伴奏发出声音.\\

[\textbf{发声练习31}] 以假声方式分别以Mi,Fa\#,So\#,Fa\#,Mi音发``は"音各2次, 每次1拍. 跟随<音轨31>重复4次.\par
\qquad 尝试每半音进行音高的提升, 跟随伴奏发出声音.\\

[\textbf{发声练习32}] 以假声方式分别以Mi,Fa\#,So\#,Fa\#,Mi音发``は"音各4次, 每次1拍. 跟随<音轨32>重复4次.\par
\qquad 尝试每半音进行音高的提升, 跟随伴奏发出声音.\\

[\textbf{发声练习33}] 以假声方式分别以Mi,Fa\#,So\#,Fa\#,Mi音拉长发``は"音. 跟随<音轨33>重复4次.\par
\qquad 尝试每半音进行音高的提升, 跟随伴奏发出声音.\\

[\textbf{发声练习34}] 以假声方式分别以Mi,Fa\#,So\#,La,Si,Do\#,Re音发``ふ"音每个9拍. 跟随<音轨34>重复4次.\par
\qquad 尝试每半音进行音高的提升, 跟随伴奏发出声音.\\

[\textbf{发声练习35}] 以假声方式变调分别以Mi,Fa\#-So\#,Fa\#-Mi,Fa\#-So\#,F\#-Mi音发``ひ"音每个9拍. 跟随<音轨35>重复4次.\par
\qquad 尝试每半音进行音高的提升, 跟随伴奏发出声音.\par
\qquad 1. 如果感受到嗓子不适请休息至少30秒再继续练习.\par
\qquad 2. 如果感受到头晕脑胀或眼睛不适, 请稍事休息进行数个深呼吸, 待缓解缺氧状态后再进行练习.\\

[\textbf{发声练习36}] 以假声方式Mi音发''られりろ"各半拍, 并以Mi到Si的音阶各为1组进行练习. 跟随<音轨36>重复4次.\par
\qquad 尝试每半音进行音高的提升, 跟随伴奏发出声音.\\

\clearpage

[\textbf{发声练习37}] 以假声方式Mi音连续发''かけきこ", 并以Do到Si的音阶各为1组进行练习. 跟随<音轨37>重复4次.\par
\qquad 尝试每半音进行音高的提升, 跟随伴奏发出声音.\par
\qquad 注意: 增加了音的数量后, 假声集中可能变的困难, 这里不应该跳过而应反复练习.\\

[\textbf{发声练习38}] 以假声方式Mi,Fa\#,So\#,La,Si,La,So\#,Fa\#音发''させしすせそさそ"各半拍,. 跟随<音轨38>重复4次.\par
\qquad 尝试每半音进行音高的提升, 跟随伴奏发出声音.\par
\qquad 注意: 如果发不出声音请不要勉强, 稍作休息再试.\\

[\textbf{发声练习39}] 以假声方式Mi,Fa\#,So\#,La,Si,La,So\#,Fa\#音发''たてちつとたと"各半拍. 跟随<音轨39>重复4次.\par
\qquad 尝试每半音进行音高的提升, 跟随伴奏发出声音.\par
\qquad 注意: 发声练习27到发声练习39需要每天反复练习.\\

[\textbf{确认发声练习的成果}] 使用录音器材, 记录以假声方式发出的``为什么会这样动听呢"语句.\par
\qquad 注意: 不应强迫自己发出较大的声音, 可尝试从私语逐渐提高音量. 进行前需用手确认喉结两侧肌肉柔软放松. 自始至终使用腹式呼吸.\\

[\textbf{发声练习40}] 尝试像<音轨40>一样发出''あ"音, 体验混声.\\

[\textbf{发声练习41}] 以``あ"音从本声过渡到假声, 逐渐拉长过渡时间. 跟随<音轨41>反复练习.\par
\qquad 无法发出混声时, 可以尝试在本音中掺入少许呼吸.\\

[\textbf{自行练习1}] 尝试将''かきくけこ"以浊音方式``がぎぐげご"发出.\par
\qquad 当喉结上提时将发出鼻浊音, 需要学会暂时不发出鼻浊音.\\

[\textbf{鼻浊音的适用例子}] \par
\qquad 1. \mincho ''学校(がっこう)に行(い)くのが楽(たの)しみだね"\rmfamily 中第一个"が"以浊音发声, 第二个''が"以鼻浊音发声将显得更为平滑自然.\par
\qquad 2. \mincho ``月曜日(げつようび)がこんなに待(ま)ち遠(どお)しいのは初(はじ)めて".\rmfamily \par
\qquad 3. \mincho ''痩せすぎて、ガリガリになっちゃうよ".\rmfamily\\

\clearpage

\begin{center}
 \textbf{5.发不出声音的时候}
\end{center}

[\textbf{声带的长度}] 男性17-23mm, 女性12-17mm.\\

[\textbf{声音嘶哑的可能原因}] \par
\qquad 1. 高强度使用嗓子: 当嗓子疼痛时继续进行练习会给声带造成损伤, 声音也会变得嘶哑. 久而久之甚至可能引起声带小结;\par
\qquad 2. 摄入烟酒: 想要练习成果的话, 饮酒要适量, 不能吸烟;\par
\qquad 3. 睡眠不足: 应确保每天睡眠6小时及以上, 疲劳也会造成声带损伤.\\

[\textbf{无法发声对策1}] 在全身放松的情况下不用力地发声, 尤其不要在嗓子和腹部用力. 在使力发声后全身放松.\\

[\textbf{无法发声对策2}] 若无法自主放松, 可以尝试只发出叹息样的``は"和''ふ"音. 当疲劳或寒冷时, 请尝试进入温暖的浴池发声.\\

[\textbf{无法发声对策3}] 尝试在发声过程中摇摆身体. 以腰为中心, 身体左右旋转. 尝试发出``ほ"和''は"音. 当注意力转移到腰部后, 声带集中用力的问题也将能够解决.\\

[\textbf{无法发声对策4}] 无法发出混声时, 尝试想象声音从眉间发出.\\

[\textbf{无法发声对策5}] 发出的混声接近本声或假声时, 尝试大哈欠以放松声带, 并以腹式呼吸放松腹部.\\

[\textbf{无法发声对策6}] 无法控制混声音高时, 尝试发出``ん"音, 并使鼻尖下方震动发痒.\\

[\textbf{无法发声对策7}] 重复练习.\\

[\textbf{声道伸展的方法}] 打开下颚, 对喉咙深处进行扩张. 保持姿势以''お"的口型发声. 此时应尽量可以从镜子里看到嗓子深处的样子. 再依照五十音图逐个练习.\\

[\textbf{坐着练习的姿势}]\par
\qquad 1. 背部稍稍离开椅子, 避免驼背;\par
\qquad 2. 腰部伸直, 避免弯腰;\par
\qquad 3. 腰部不要过度伸展, 避免胸部突出影响膈肌运动.\\

\clearpage

\begin{center}
 \textbf{6.正式的女声练习}
\end{center}

[\textbf{Melanie训练法}] Melanie Phillips提出的男性学习女性声音的方法. 前5章已经进行了完整的训练.\\

[\textbf{区分声音性别的要素}]\par
\qquad 1. 频率: 男女声音频率差别仅有$\frac{1}{2}$个八度左右, 但只提高声音频率并不会成为女性的声音;\par
\qquad 2. 音质和音色: 次要频率分量;\par
\qquad 3. 动态范围: 在日语中, 女性的日常用语有句尾提升等抑扬顿挫的形式;\par
\qquad 4. 特定词语的发音: 汉语不常见;\par
\qquad 5. 词汇语法: 不同性别常用词汇的差别, 汉语不常见.\\

[\textbf{Melanie发声检查}] 将右手食指放在喉结上方, 左手食指放在喉结下方, 以平时说话的方式随意讲话.\par
\qquad 1. 若上下几乎一样震动则声音应接近普通的男性;\par
\qquad 2. 若上下几乎都不震动的话, 则声音较为接近普通的女性;\par
\qquad 3. 若只有下部震动, 则声音可以非常接近女性.\\

[\textbf{发声的要点}] 漱口时, 吸入空气后暂停呼吸, 尝试抬起头让漱口水中发出``がらがらがら"的气泡声. 此时喉咙中共鸣的地方和以较低假声发声时的位置相同. 练习时尝试让这里震动发声.\\

\clearpage

\begin{center}
 \textbf{7.实录篇}
\end{center}

[\textbf{练习的场地选择}]\par
\qquad 1. 隔音效果好的室内: 注意避免打扰到邻居;\par
\qquad 2. 在卡拉OK练习: 注意避免在练习时吃东西或喝饮料;\par
\qquad 3. 嗓音训练教室.\\

[\textbf{国内已知的嗓音训练教室}]\par
\qquad 1. 上海复旦大学眼耳鼻喉科医院, 耳鼻喉科, 陈臻医生;\par
\qquad 2. 首都医科大学附属北京友谊医院, 耳鼻喉科, 李革临和侯倩医生;\par
\qquad 3. 北京大学第三医院, 耳鼻喉科, 闫燕(未确认).\\

[\textbf{录音器材}]\par
\qquad 1. 智能手机: 最为方便, 容易获取;\par
\qquad 2. 动圈式麦克风: 价格便宜, 动态范围足够;\par
\qquad 3. 电容式麦克风: 能够拾取声音细节, 但价格昂贵.\\

[\textbf{录音软件}] Linux: Audio Recorder 或 Audacity.\\

[\textbf{喉结的定位}] 不要在发出假声时一边发声一边上提喉结, 应有意识地提高喉结的位置. 尝试练习''ほんっと、いいてんき"(真的呢, 这么好的天气).\\

[\textbf{检查自己的声音}] 听自己的声音对于大多数人来说都是痛苦的, 这是正常的反应. 但请突破自己, 通过不断录制和重放, 仔细寻找自己声音的改善点.\\

[\textbf{声音需要改善的地方}]\par
\qquad 1. 声音太粗: 反复练习第2到第5章.\par
\qquad 2. 发出了奇怪的声音:\par
\qquad \qquad (1). 发出青蛙一样的声音: 尝试放松声带, 不需要勒紧嗓子发音;\par
\qquad \qquad (2). 嗓子被勒紧: 尝试放松嗓子, 重复之前的声道扩张训练;\par
\qquad \qquad (3). 无法放松: 进行足够的热声, 重复第1章的放松训练. 不要让下颚过于向下, 这样会不自觉地给声带施力;\par

\clearpage

\qquad 3. 发出很重的鼻音:\par
\qquad \qquad (1). 尝试捏住鼻子发声, 并记住此时的发声状态;\par
\qquad \qquad (2). 尝试降低声音的频率;\par
\qquad \qquad (3). 降低音量, 这是比较消极的解决方法.\par
\qquad 4. 舌头没法灵活地活动: 保持喉结上提, 重复之前的舌头训练. 稍稍抬头也能稍稍环节舌骨受到的压力.\\

[\textbf{常见的动态添加方式}]\par
\qquad 1. 句尾上提;\par
\qquad 2. 延长句尾;\par
\qquad 3. 增加附和的语气词;\par
\qquad 4. 夸张地表达.\\

[\textbf{演技练习1}] 尝试扮演角色: 快活的女子高中生. 跟随<音轨42>反复练习.\par
\qquad 1. \mincho おはよっ いい天気(てんき)だね.\rmfamily (早上好 真是个好天气呢)\par
\qquad 2. \mincho そんなこっちゃ、女の子(おんなのこ)にモテないぞっ. \rmfamily (这样的话, 可不受女孩子欢迎哦)\par
\qquad 3. \mincho いっ、いきなりそんな、可愛(かわい)いとか言(い)われても...\rmfamily (忽, 忽然这样说可爱的话, 但...)\par
\qquad 4. \mincho かっ、勘違い(かんちがい)しないでよね!\rmfamily (可, 可不要误会了啊!)\\

[\textbf{演技练习2}] 尝试扮演角色: 沉稳的少女. 跟随<音轨43>反复练习.\par
\qquad 1. \mincho お早(はよ)うございます。あれかれ受験(じゅけん)勉強(べんきょう)はかどりましたか?\rmfamily (早上好. 从那之后为考试复习了吗?)\par
\qquad 2. \mincho はう...そんなに見(み)ないでくださいよぉ.\rmfamily (哈...请不要那样看着哟)\par
\qquad 3. \mincho こういう場所(ばしょ)って初(はじ)めてなんです。とっても楽しみです.\rmfamily (第一次到这样的地方, 太高兴了)\par
\qquad 4. \mincho そういうのって、いけないと思(お)うんです.\rmfamily (那样的事情, 我认为是不可以的)\\

[\textbf{演技练习3}] 尝试扮演角色: 有些疲惫的阿姨. 跟随<音轨44>反复练习.\par
\qquad 1. \mincho そうだねぇ、あたしにもそんな頃(ごろ)があったかもねぇ.\rmfamily (对啊, 我也有那样的时候呢)\par
\qquad 2. \mincho あたしにはもう無理(むり)だってば。他(ほか)をたってよ.\rmfamily (我已經不行了啊, 做点别的吧)\par

\clearpage

\qquad 3. \mincho その気(き)にさせといて、どうせ本気(ほんき)じゃないんでしょ?\rmfamily (即使这样也不是真心的对吧)\par
\qquad 4. \mincho さあて。もうひと花(はな)咲(さ)かせますか?\rmfamily (那么, 能让另一朵花也开吗)\\

[\textbf{演技练习4}] 尝试扮演角色: 可爱的猫. 跟随<音轨45>反复练习.\par
\qquad 1. \mincho やったにゃ カツオブシだにゃ うれしいにゃ.\rmfamily (太棒了, 是柴鱼干, 好高兴)\par
\qquad 2. \mincho そんなに好きなら、種族なんて関係ないにゃ.\rmfamily (这么喜欢的话和种族也没什么关系呀)\par
\qquad 3. \mincho ひっかいてやるにゃ! こっちくるなにゃ!\rmfamily (挠你哦! 别过来!)\par
\qquad 4. \mincho ご主人さまぁ、お腹減ったにゃ... \rmfamily (主人, 肚子饿了呢)\\

\clearpage

\begin{center}
 \textbf{数据记录纸}
\end{center}

[\textbf{基础频率}] \\

[\textbf{最低频率}] \\

[\textbf{最高频率}] \\

[\textbf{音域}] \\

[\textbf{练习记录}]\\

\end{document}
